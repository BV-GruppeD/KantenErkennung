\documentclass[12pt]{beamer}
\usepackage[utf8]{inputenc}
\usepackage[T1]{fontenc}
\usepackage{lmodern}
\usepackage{ngerman}
\usepackage{amsmath}
\usepackage{amsfonts}
\usepackage{amssymb}
\usepackage{graphicx}
\usepackage{pgfplots}
\usepackage{geometry}
\usepackage{fancyhdr}
\usepackage{lastpage}
\usepackage{float}
\usepackage{tikz}
\usepackage[european,smartlabels,siunitx]{circuitikz}
\usetikzlibrary{calc,positioning}

\pgfdeclareimage[width = 2cm]{meinlogo}{Logo.png}

\usetheme{CambridgeUS}
\usecolortheme{seagull}

\begin{document}
	\author{Gruppe D}
	\title{Kantenerkennung}
	\subtitle{Projekt 1}
	\logo{\pgfuseimage{meinlogo}}
	\institute[HS OWL]{Hochschule Ostwestfalen Lippe}
	\date{31.10.2018}
	\subject{was das}
	%\setbeamercovered{transparent}
	\setbeamertemplate{navigation symbols}{}
	
	
\begin{frame}
	\titlepage
\end{frame}

\begin{frame}
	\frametitle{Inhalt}
	\tableofcontents	
\end{frame}

\section{Einleitung}
	\begin{frame}
		\frametitle{Problemstellung}
		\begin{itemize}
			\item erster
			\item zweiter
			\item dritter
			\item vierter
			\item fünfter der später kommt		
		\end{itemize}
		
	\end{frame}

\section{Lösungsweg}
\begin{frame}
	\frametitle{Punktoperation}
	\begin{block}{Schwellwerterkennung}	
		\begin{eqnarray*}
			f_{treshold}=
			\begin{cases}
			a_0   			& \text{für }i_N(t) > i_{NV}\\
			a_1        		& \text{für }i_N(t) \leq i_{NV} \\
			a_2        		& \text{für }i_N(t) \leq i_{NV}
			\end{cases}
		\end{eqnarray*}
	\end{block}
\end{frame}

 \section{Fazit}
	\begin{frame}
		\frametitle{Fazit}
		\begin{itemize}
			\item erster
			\item zweiter
			\item dritter
			\item vierter
			\item fünfter der später kommt		
		\end{itemize}
	\end{frame}

\section*{}


\begin{frame}
	\frametitle{References}
	\Large{Vielen Dank für eure Aufmerksamkeit}
	
	\footnotesize{
		\begin{thebibliography}{99} % Beamer does not support BibTeX so references must be inserted manually as below
			\bibitem[Burger, Burge, 2015]{p1} Wilhelm Burger, Mark James Burge (2015)
			\newblock Digitale Bildverarbeitung: Eine algorithmische Einführung mit Java
			%\newblock \emph{Journal Name} 12(3), 45 -- 678.
		\end{thebibliography}
	}
\end{frame}


\end{document}